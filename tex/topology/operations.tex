\section{Операции и конструкции}
\begin{defin}
	Пусть $f\colon X\to Y$ --- произвольное отображение, а на $Y$ определена топология. \textit{Начальная топология (initial topology) на $X$ относительно отображения $f$} --- это слабейшая топология, относительно которой отображение $f$ непрерывно.
\end{defin}
Несложно дать явное задание этой топологии, стоит только вспомнить определение непрерывности: необходимо и достаточно, чтобы были открыты все множества вида $f^{-1}(U)$, где $U$ открыто в $Y$, то есть $\sigma=\{f^{-1}(U)|U\text{ --- открыто в Y}\}$ --- предбаза начальной топологии. Эта терминология не может считаться устоявшейся в русской литературе и не встречается за пределами этого конспекта.

Двойственным понятием будет \textit{финальная топология}.
\begin{defin}
	Пусть $f\colon X\to Y$ --- произвольное отображение, а на $X$ определена топология. \textit{Финальная топология (final topology) на $Y$ относительно отображения $f$} --- это сильнейшая топология, относительно которой отображение $f$ непрерывно.
\end{defin}
Зададим явно и эту топологию: множество $U\subset Y$ открыто тогда и только тогда, когда его прообраз $f^{-1}(U)$ открыт в $X$.
\begin{quest}
	Что меняется, когда вместо одного отображения $f\colon X\to Y$ рассматривается семейства отображений $f_i\colon X\to Y_i$ и $f_i\colon X_i\to Y$ в определениях начальной и финальной топологии соответственно?
\end{quest}
Приведём классические примеры этих топологий.

\textbf{Индуцированная топология.} Пусть $A\subset X$. Рассмотрим \textit{отображение включения} $i\colon A\incl{} X$, $i(a)=a$. Если на $X$ есть топология, то, чтобы задать её на $A$, можно потребовать, чтобы все открытые в $A$ множества имели вид $U\cap A$, где $U$ открыто в $X$. Элементарно проверяется, что такие множества действительно образуют топологию на $A$ и что $i$ оказывается непрерывным. Более того, это слабейшая топология, относительно которой включение $i$ непрерывно.

\textbf{Фактортопология.} Пусть $X$ --- топологическое пространство, на котором определено \textit{отношение эквивалентности} $\sim$. Рассмотрим множество $X/\sim$ и каноническую \textit{проекцию} $\pi\colon X\to X/\sim$, $x\to[x]$. Множество $X/\sim$ можно снабдить топологией, потребовав, чтобы $U\subset X/\sim$ было открыто тогда и только тогда, когда $\pi^{-1}(U)$ открыт в $X$.

Полезный пример фактортопологии --- \textit{стягивание}. Пусть $A\subset X$, а точки $x,y\in X$ связаны отношением эквивалентности $\sim$ тогда и только тогда, когда $x,y\in A$. Тогда говорят, что $X/A$ получено стягиванием подпространства $A$ в точку.

\textbf{Топология произведения.} Пусть $X$, $Y$ --- топологические пространства. Прямое произведение $X\times Y$ можно наделить естественной топологией, потребовав, чтобы канонические проекции $\pi_X\colon X\times Y\to X$ и $\pi_Y\colon X\times Y\to Y$ были непрерывны. Аналогично топология определяется для произведения произвольного семейства пространств $\{X_{\alpha}\}_{\alpha\in\mathcal{A}}$, тогда требуется, чтобы все $\pi_{\beta}\colon\prod X_{\alpha}\to X_{\beta}$ были непрерывны. Она называется \textit{тихоновской}. В случае конечного произведения её задание тривиально: множества вида $U\times V$, где $U\subset X$ и $V\subset Y$ открыты, образуют базу в $X\times Y$.

Если на $X\times Y$ ввести тихоновскую топологию, это пространство станет категорным произведением пространств $X$ и $Y$.

Классический результат --- теорема Тихонова.
\begin{theorem}
	Если все $X_{\alpha}$, $\alpha\in\mathcal{A}$, компактны. Тогда $\prod\limits_{\alpha\in\mathcal{A}}$ $X_{\alpha}$ компактно.
\end{theorem}
\begin{proof}
Докажем теорему для произведения двух пространств, тогда то же будет верно для любого конечного произведения.

Утверждение теоремы в случае бесконечного числа множителей эквивалентно аксиоме выбора.
\end{proof}

\textbf{Несвязное объединение}. Теоретико-множественное несвязное объединение определяется следующим образом. Пусть $A$ и $B$ множества, возможно, имеющие ненулевое пересечение. Мы хотим, чтобы в несвязное объединение элементы из пересечения вошли ``дважды'': как элементы $A$ и как элементы $B$. Положим $A\sqcup B=\{(a,0), (b,1)|a\in A, b\in B\}$. Вместе с этим определяются вложения $i_A\colon A\incl{} A\sqcup B$, $a\mapsto (a,0)$. Эта конструкция является категорной суммой в категории множеств.

Если $X$ и $Y$ --- топологические пространства, то $A\sqcup B$ наделяется естественной топологией и оказывается суммой в топологической категории.

\textbf{Цилиндр.} Пусть $X$ --- пространство. \textit{Цилиндром над} $X$ называют произведение $X\times I$.

\textbf{Конус.} Если стянуть цилиндр над $X$ по верхнему основанию, то получится \textit{конус над} $X$:
	\begin{equation*}
		CX=(X\times I)/(X\times\{1\}).
	\end{equation*}

\textbf{Надстройка.} Если же у конуса над $X$ нижнее основание, то получится \textit{надстройка над} $X$:
	\begin{equation*}
		\Sigma X=CX/(X\times\{0\}).
	\end{equation*}
Эквивалентно, надстройка над $X$ --- это два конуса над $X$, склеенные по основаниям.

\begin{prop}(экспоненциальный закон)
	Имеет место естественная биекция $\Phi\colon C(X\times Y, Z)\to C(X,C(Y,Z))$ (в других обозначениях $\Phi\colon Z^{X\times Y}\to (Z^Y)^X$), ставящее в соответствие отображению $f\colon X\times Y\to Z$ отображение $\Phi f\colon X\to C(Y,Z)$, действующее по правилу $(\Phi f)(x)(y)=f(x,y)$. Если $X$ хаусдорфово, а $Y$ хаусдорфово и локально компактно, то $\Phi$ --- гомеоморфизм.
\end{prop}
\begin{prop}
	Пусть $X$ хаусдорфово. Тогда имеет место естественный по $X$ и $Y$ изоморфизм $C(\sigma X, Y)\to C(X,\Omega Y)$. 
\end{prop}
\begin{proof}
	Согласно экспоненциальному закону, имеет место изоморфизм
		\begin{equation*}
			C(X\times I,Y)\to C(X, C(I,X)).
		\end{equation*}
\end{proof}