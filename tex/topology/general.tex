\section{Общая топология}
Если вы захотите изучить \textit{непрерывность} в самом общем случае, вы придёте к понятиям \textit{окрестности}, \textit{топологии} и \textit{непрерывного отображения}.

В параграфе 1 мы обсудим, каким образом можно мыслить, чтобы рассмотрение этих вещей выглядело сколько-то естественно. Далее будет рассматриваться топологическая машинерия и её пересечение с категорными явлениями, которая закончится фундаментальной группой и её классическими приложениями. 

Всюду в этой главе под пространством мы понимаем топологическое пространство, а под отображениями --- непрерывные отображения между ними.

\subsection{Топологическое пространство и непрерывные отображения}
Вспомним школьное определение непрерывности функции в точке, которое приводится в любом учебнике по введению в матанализ.
\begin{defin}
	Функция $f\colon\RR\to\RR$ непрерывна в точке $x_0\in\RR$, если $\displaystyle\lim_{x\to x_0} f(x)=f(x_0)$.
\end{defin}
Школьнику оно может быть объяснено, например, так: представим груз, подвешенный на нитке. Если на груз положить сверху дополнительный грузик, незначительный по массе в сравнении с самим грузом, то нитка незаметно для нашего глаза растянется. Мы говорим, что длина нити \textit{непрерывно зависит} от подвешенной массы. Если положить слишком большой грузик, то нитка растянется слишком сильно и порвётся...

Посмотрим, как это определение обобщить. Для начала, его можно развернуть, используя $\varepsilon$-$\delta$-формализм:
	\begin{equation}
		\forall\varepsilon>0\ \exists\delta\such\forall x\such |x-x_0|<\delta\ |f(x)-f(x_0)|<\varepsilon.
	\end{equation}
Первый возможный шаг --- это замена модулей разности на \textit{расстояния} между точками.
\begin{defin}
	\textit{Метрикой} или расстоянием на множестве $X$ называют функцию $\rho\colon X\times X\to\RR$, удовлетворяющую аксиомам:
		\begin{enumerate}
			\item $\forall x,y\in X\ \rho(x,y)\geqslant 0$;
			\item $\forall x,y\in X\ \rho(x,y)=0\Leftrightarrow x=y$;
			\item $\forall x,y\in X\ \rho(x,y)=\rho(y,x)$;
			\item $\forall x,y,z\in X\ \rho(x,z)\leqslant\rho(x,y)+\rho(y,z)$.
		\end{enumerate}
Множество $X$ с введёной на ней метрикой $\rho$ называют \textit{метрическим пространством} и пишут $(X,\rho)$.
\end{defin}

Модуль разности $d(x,y)=|x-y|$ удовлетворяет всем четырём аксиомам. В сущности, при доказательстве свойств предела --- арифметических и прочих --- ничем, кроме этих свойств модуля, мы не пользуемся. Иными словами, о конкретном виде метрики можно не думать, нужны лишь её свойства. Теперь мы можем перенести определение непрерывности в точке с числовых функций на отображения между множествами, на которых введена метрика.
\begin{defin}
	Пусть $(X,\rho)$, $(Y,d)$ --- метрические пространства. Функция $f\colon X\to Y$ непрерывна в точке $x_0\in X$, если
	\begin{equation}
		\forall\varepsilon>0\ \exists\delta\such\forall x\such \rho(x,x_0)<\delta\ d(f(x),f(x_0))<\varepsilon.
	\end{equation}
\end{defin}
Можно пойти дальше. В метрическом пространстве $(X,\rho)$ определим \textit{$\delta$-окрестность точки $x$} как множество $U_{\delta}(x)=\{y\in X\ |\ \rho(x,y)<\delta\}$, это шар радиуса $\delta$ с центром в точке $x$. Определение непрерывности тогда можно записать как
	\begin{equation}
		\forall\varepsilon>0\ \exists\delta\such\forall x\in U_{\delta}(x_0)\ f(x)\in U_{\varepsilon}(f(x_0)),
	\end{equation}
или, ещё короче,
	\begin{equation}
		\forall\varepsilon>0\ \exists\delta\such f(U_{\delta}(x_0))\subset U_{\varepsilon}(f(x_0)).
	\end{equation}
Повторим фокус снова: забудем про внутреннее устройство окрестности. Будем теперь считать, что каждой точке $x$ множества $X$ приписано семейство $\mathcal{O}(x)$ подмножеств множества $X$, называемых окрестностями этой точки, свойства которых мы потом отдельно выделим. Наиболее общо, определение непрерывности в точке теперь выглядит так, если мы предполагаем, что на множествах $X$ и $Y$ введены эти системы окрестностей.
\begin{defin}
	Функция $f\colon X\to Y$ непрерывна в точке $x\in X$, если
	\begin{equation}
		\forall U\in\mathcal{O}(f(x))\ \exists V\in\mathcal{O}(x)\such f(V)\subset U.
	\end{equation}
\end{defin}
Что мы ожидаем от окрестностей точки? Выделим следующие естественные их свойства.
\begin{enumerate}
	\item если $U, V\in\mathcal{O}(x)$, то $U\cap V\in\mathcal{O}(x)$;
	\item если $U, V\in\mathcal{O}(x)$, то $U\cup V\in\mathcal{O}(x)$;
\end{enumerate}

Эта система аксиом, хоть и выглядит естественной, нигде не применяется. Она эквивалентна традиционному определению топологического пространства.

Итак, топология на множестве --- необходимая структура для определения непрерывных отображений.
\begin{defin}\label{defin:topol}
	Пусть $X$ --- множество. Топологией на $X$ назовём семейство $\tau$ подмножеств множества $X$, удовлетворяющее следующим требованиям:
	\begin{enumerate}
		\item $\emptyset\in\tau$, $X\in\tau$;
		\item для всякого конечного набора подмножеств $\{X_i\}_{i=1}^n\subset\tau$ верно $\bigcap\limits_{i=1}^n X_i\in\tau$;
		\item для всякого набора множеств $\{X_{\alpha}\}_{\alpha\in A}\subset\tau$ верно $\bigcup\limits_{\alpha\in A} X_{\alpha}\in\tau$.
	\end{enumerate}
	Если $U\in\tau$, то $U$ называют \textit{открытым} подмножеством $X$ (в этой топологии). Множество $V\subset X$ замкнуто, если его дополнение открыто.
\end{defin}
Поясним то, что написано. С первым пунктом определения~\ref{defin:topol} всё должно быть более-менее ясно: пустое множество и всё множество будем считать открытыми. Согласно второму пункту, \textit{пересечение} всякого \textit{конечного} семейства открытых множеств снова открыто. Третий же пункт утверждает, что \textit{объединение} \textit{любого} семейства открытых множеств открыто.

Из определения ясно, что задание на множестве замкнутых подмножеств эквивалентно введению топологии. Для них, как это следует из правил де Моргана, любые \textit{конечные} объединения замкнуты и \textit{произвольные} пересечения замкнуты.
\begin{defin}
	Пусть $U\subset X$ открыто в $X$, а $x\in U$. Тогда $U$ --- \textit{окрестность} точки $x$.
\end{defin}

Покажем, что система аксиом окрестностей точки эквивалентна заданию топологии на множестве. 

Приведём первые примеры.
\begin{enumerate}
	\item На всяком множестве $X$ можно рассматривать топологии $\{\varnothing, X\}$ и $2^X$. Последнюю также называют \textit{дискретной}: одноточечные множества открыты и замкнуты и каждую точку можно отделить от всех остальных окрестностью, состоящей из её одной.
	\item На непустом множестве $X$ $\{\varnothing, \{x\}, X\}$ --- топология.
	\item На прямой $X=\RR$ открытыми объявляются все не более чем счёные объединения интервалов вида $(a,b)$, где $a,b\in\RR\cup\{\pm\infty\}$. На эту топологию будем ссылаться как на обычную или стандартную.
	\item На $\RR$ можно ввести и другую топологию: замкнутыми пусть будут все конечные множества и $X$.
	\item На отрезке $I=[0,1]$ открытыми объявляются пересечения открытых в стандартной топологии на $\RR$ множеств с $I$.
\end{enumerate}
\begin{defin}
	Отображение $f\colon X\to Y$ непрерывно, если прообраз $f^{-1}(U)$ всякого открытого в $Y$ множества $U$ открыт в $X$.
\end{defin}
Эквивалентно, отображение непрерывно, если прообраз замкнутого множества замкнут.
\begin{prop}
	Отображение $f\colon X\to Y$ непрерывно тогда и только тогда, когда оно непрерывно в каждой точке $x\in X$.
\end{prop}
\begin{proof}
	\textbf{Необходимость.} Пусть $f$ непрерывна, а $x\in X$. Рассмотрим произвольную окрестность $V\subset Y$ точки $f(x)$, тогда $U=f^{-1}(V)$ открыто и $x\in U$.

	\textbf{Достаточность.} Пусть $f$ непрерывна в каждой точке $x\in X$. Рассмотрим произвольное открытое множество $V\subset Y$. Для каждого $y\in V$ множество $V$ --- его окрестность, а для каждого $x\in X$ такого, что $f(x)\in V$, найдётся его окрестность $U(x)$ такая, что $f(U(x))\subset V$, значит, $U(x)\subset f^{-1}(V)$. Имеем:
		\begin{equation*}
			\bigcup_{x\in X,\newline f(x)\in V} U(x)\subset f^{-1}(V)\subset\bigcup_{x\in X, f(x)\in V}\{x\}\subset\bigcup_{x\in X, f(x)\in V} U(x).
		\end{equation*}
	Таким образом, прообраз $f^{-1}(V)=\bigcup_{x\in X,\newline f(x)\in V} U(x)$ открыт как объединение открытых множеств.
\end{proof}

Класс топологических пространств вместе с непрерывными отображениями в качестве морфизмов составляют \textit{топологическую категорию} $\mathcal{T}op$. Конечным объектом в этой категории будет одноточечное пространство $\{*\}$, так как в него существует единственное отображение. О суммах и произведениях в этой категории речь пойдёт в следующем параграфе. 

Изоморфизмы в категории $\mathcal{T}op$ называют \textit{гомеоморфизмами}. Так, непрерывное отображение $f\colon X\to Y$ --- гомеоморфизм, если оно биективно и обратное отображение $f^{-1}\colon Y\to X$ непрерывно. Если для пространств $X$ и $Y$ найдётся гомеоморфизм $f\colon X\to Y$, то они называются \textit{гомеоморфными}, пишут $X\equiv Y$. В топологии пространства изучаются с точностью до гомеоморфизма. Так, единичный евклидов шар $D^n=B_1(0)\subset\RR^n$ (он же $n$-мерный диск) гомеоморфен $n$-мерному кубу $[-1;1]^n$. Вообще говоря, всякое односвязное тело в $\RR^n$ гомеоморфно $n$-мерному диску.

Одной биективности недостаточно для гомеоморфности: так, например, если $S^1=\{e^{2\pi it}|\varphi\in[0,1)\}$, то $e^{2\pi it}\mapsto t$ не является гомеоморфизмом, что и естественно: нельзя превратить окружность в интервал, нигде её не разорвав.

Топологии на множестве $X$ можно сравнивать: если $\tau_1$ и $\tau_2$ --- топологии на $X$, то говорят, что $\tau_1$ \textit{сильнее (тоньше)} $\tau_2$ и что $\tau_2$ \textit{слабее (грубее)} $\tau_1$, если $\tau_2\subset\tau_1$. Если никакое включение не выполняется, то топологии не сравнивают. Так, топология $\{\varnothing, X\}$ --- слабейшая топология на $X$, а $2^X$ --- сильнейшая.

В задании топологий на множестве важны понятия базы и предбазы.
\begin{defin}
	Семейство подмножеств $\beta\subset\tau$ пространства $X$ --- \textit{база топологии $\tau$ на $X$}, если всякое открытое множество представимо в виде (произвольного) объединения открытых множеств из $\beta$.
\end{defin}
\begin{defin}
	Семейство подмножеств $\sigma\subset\tau$ пространства $X$, конечные пересечения множеств которого образуют базу топологии $\tau$ на $X$, --- это \textit{предбаза} топологии $\tau$.
\end{defin}
Предбаза --- это набор множеств, ``порождающий'' топологию. Пусть имеется семейство $\sigma$ подмножеств множества $X$. Мы хотим, чтобы эти множества были открыты в некоторой топологии, причём желательно, чтобы она не содержала ``лишних'' открытых множеств. Тогда из этого набора нужно получить всевозможные конечные пересечения входящих в него множеств, а то, что получилось, любым образом прообъединять. Эквивалентно, рассмотрим множество $\mathcal{T}(\sigma)$ всех топологий на $X$, которые содержат в себе семейство $\sigma$. Тогда $\sigma$ --- предбаза топологии $\tau_{\sigma}=\bigcap\limits_{\tau\in\mathcal{T}(\sigma)}\tau$.

\begin{defin}
	Точка $x\in X$ --- \textit{точка прикосновения $X$}, если каждая её окрестность $U$ содержит ещё какую-то точку из $X$, то есть $(U\setminus{X})\cap X\neq\varnothing$.
\end{defin}
\begin{prop}
	Множество $A\subset X$ замкнуто тогда и только тогда, когда содержит все свои точки прикосновения.
\end{prop}
\begin{proof}
	\textbf{Необходимость.} Пусть $A$ замкнуто, а $x\in X$ --- точка прикосновения $A$. Предположим, $x\in X\setminus A$. По определению любая окрестность $U$ точки $x$ содержит точку, отличную от $x$ и входящую в $A$. Поскольку $X\setminus A$ открыто и является окрестностью точки $x$, оно пересекает $A$. Противоречие.

	\textbf{Достаточность.} Пусть $A$ содержит все свои точки прикосновения. Возьмём $x\in X\setminus A$, тогда, раз $x$ --- не точка прикосновения $A$, найдётся её окрестность $U(x)$ такая, что $A\cap U(x)=\varnothing$. Имеем $\bigcap_{x\in X\setminus A} U(x)\subset X\setminus A\subset\bigcap_{x\in X\setminus A} \{x\}\subset\bigcap_{x\in X\setminus A} U(x)$. Значит, $X\setminus A=\bigcap_{x\in X\setminus A} U(x)$.
\end{proof}

\subsection{Отделимость}
\begin{defin}
	Пространство $X$ хаусдорфово, если для всякой пары точек $x,y\in X$, $x\neq y$, найдутся их окрестности $U$,$V$ такие, что $U\cap V=\varnothing$.
\end{defin}

\subsection{Сходимость, последовательности и направленности}
Введения топологии достаточно, чтобы определить \textit{сходимость}.
\begin{defin}
	Пусть $\{x_n\}_{n=1}^{\infty}\subset X$. Тогда $x_i\rightarrow x$, если для всякой окрестности $U$ точки $x$ найдётся $n\in\NN$ такой, что для всех $m\geqslant n$ $x_m\in U$.
\end{defin}

\subsection{Компактность}
Без компактности в топологии абсурд и коррупция.
\begin{defin}
	Говорят, что семейство $\mathcal{U}=\{U_{\alpha}\}_{\alpha\in\mathcal{A}}\subset 2^X$ \textit{покрывает} (или является покрытием) множества $A\subset X$, если $A\subset\bigcup\limits_{\alpha\in\mathcal{A}} U_{\alpha}$. Если $\mathcal{U}'\subset\mathcal{U}$ и $\mathcal{U}'$ покрывает $A$, то $\mathcal{U}'$ --- подпокрытие покрытия $\mathcal{U}$.
\end{defin}
\begin{defin}
	Пусть $K\subset X$ таков, что из любого покрытия $\mathcal{U}=\{U_{\alpha}\}_{\alpha\in\mathcal{A}}\subset 2^X$, состоящего из открытых подмножеств пространства $X$, можно выделить конечное подпокрытие $\mathcal{U}'=\{U_{\alpha_i}\}_{i=1}^n$. Тогда пространство $K$ называют \textit{компактом}.
\end{defin}
\begin{prop}
	Пусть $X$ компактно, а $f\colon X\to Y$ непрерывно. Тогда $f(X)\subset Y$ компактно.
\end{prop}
\begin{proof}
	Пусть $\mathcal{U}=\{U_{\alpha}\}_{\alpha\in\mathcal{A}}$ --- открытое покрытие $f(X)$, тогда $\{f^{-1}(U_{\alpha})\}_{\alpha\in\mathcal{A}}$ --- открытое покрытие $X$. Выделим из него открытое подпокрытие $\{f^{-1}(U_{\alpha_i}\}_{i=1}^n$. Значит, $\{U_{\alpha_i}\}_{i=1}^n$ --- открытое подпокрытие покрытия $\mathcal{U}$.
\end{proof}
\begin{prop}
	Пусть $X$ компактно, а $A\subset X$ замкнуто. Тогда $A$ --- компактно.
\end{prop}
\begin{proof}
	Пусть $\mathcal{V}=\{V_i\}$ --- открытое покрытие пространства $A$, тогда $V_i=U_i\cap A$ для какого-то открытого в $X$ множества $U_i$, $i=\overline{1,n}$. Для $X$ $\mathcal{U}=\{X\setminus A\}\cup\{U_i\}$ --- открытое покрытие. Выделим из него конечное подпокрытие $\mathcal{U}'=(X\setminus A)\cup\{U_1,\ldots, U_n\}$, значит, $\mathcal{V}'=\{V_1,\ldots, V_n\}$ --- конечное подпокрытие покрытия $\mathcal{V}$.
\end{proof}
Следующее вспомогательное утверждение всплывает, например, в лемме Гейне-Бореля, классифицирующей все компактные подмножества пространства $\RR^n$.
\begin{prop}
	Компакт в хаусдорфовом пространстве замкнут.
\end{prop}

Важное для приложений свойство, вытекающее из компактности, --- \textit{секвенциальная компактность}. В матанализе мы любим изучать свойства последовательностей.
\begin{defin}
	Пространство $X$ \textit{секвенциально компактно}, если из всякой последовательности элементов из $X$ можно выделить сходящуюся подпоследовательность.
\end{defin}
\begin{prop}
	Компактное пространство секвенциально компактно.
\end{prop}
