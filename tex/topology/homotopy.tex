\section{Гомотопическая топология}
\subsection{Пространства с отмеченной точкой}
Так будем называть пары $(X,x_0)$, где $X$ --- топологическое пространство, а $x_0\in X$. Про непрерывное отображение $f\colon(X,x_0)\to (Y,y_0)$ будем говорить, что оно \textit{сохраняет отмеченную точку}, если $f(x_0)=y_0$. Пространства с отмеченной точкой вместе с отображениями, их сохраняющими, в качестве морфизмов, образуют \textit{пунктированную категорию} $\mathcal{T}op^*$.

В терминах пространств с отмеченной точкой даётся определение фундаментальной группы, высших гомотопических групп и групп гомологий. Для их нужд предыдущие конструкции прокачаем до их аналогов с отмеченными точками.

\textbf{Надстройка.} Для пространства $X$ определим
	\begin{equation*}
		\Sigma X=X\times I
	\end{equation*}

Определим \textit{букет}, являющийся суммой в категории $\mathcal{T}op^*$.
\begin{defin}
	Пусть $\{(X_{\alpha},x_{\alpha}\}_{\alpha\in\mathcal{A}}$ --- семейство пространств с отмеченными точками. Их \textit{букетом} назовём пространство
		\begin{equation*}
			\bigvee\limits_{\alpha\in\mathcal{A}}X_{\alpha}=\bigsqcup\limits_{\alpha\in\mathcal{A}}X_{\alpha}\big/\{x_\alpha\}_{\alpha\in\mathcal{A}}.
		\end{equation*}
\end{defin}
Иными словами, все пространства приклеили друг к другу по отмеченным точкам: так, букет двух окружностей --- восьмёрка.

Произведение в $\mathcal{T}op^*$ менее хитрое: отмеченной точкой в п

\subsection{Гомотопия и гомотопическая эквивалентность}
Если попытаться придать точный смысл \textit{деформации}, то получится гомотопия. Нарисуем в воображении некоторый объект, будем считать, что он был получен как непрерывный образ $f(X)$ какого-то множества $X$ в $Y$. ``Раскадрируем'' то, как объект деформируется с течением времени, получим серию картинок $f_0(X)=f(X), f_1(X),\ldots, f_n(X)\subset Y$. Эту серию можно собрать в одно отображение $H\colon X\times\{0,1,\ldots,n\}\to Y$, непрерывное по первой переменной. Пожелав, чтобы кадры менялись непрерывно, мы определим отображение $H\colon X\times I\to Y$ такое, что на каждом шаге $t\in I$ имеем некоторую непрерывную функцию $f_t(x)=H(x,t)$, причём $f_0=f$, $f_1=g$.

Изучение пространств и отображений с точностью до гомотопии важно, потому что все используемые функторы из топологической категории не чувствительны к гомотопиям. Прежде чем давать определения, докажем полезное и несложное утверждение, на которое будет удобно сослаться в дальнейшем.
\begin{lem}[о склейке]
	Пусть пространство $X$ представлено конечным объединенем замкнутых множеств $X_i$, и заданы непрерывные отображения $f_i\colon X_i\to Y$, причём если $X_{ij}=X_i\cap X_j$ непусто, то $f_i|_{X_{ij}}=f_j|_{X_{ij}}$. Тогда существует единственное непрерывное отображение $f\colon X\to Y$ такое, что $f|_{X_i}=f_i$.
\end{lem}
\begin{proof}
	Искомое отображение так и задаётся: $f(x)=f_i(x)$, если $x\in X_i$, и в силу условия леммы корректно определено. Покажем, что оно непрерывно. Пусть $M\subset Y$ замкнут, тогда
		\begin{equation*}
			f^{-1}(M)=\bigcap_i X_i\cap f^{-1}(M)=\bigcap_i f_i^{-1}(M).
		\end{equation*}
	Прообраз замкнутого множества замкнут как конечное объединение замкнутых множеств.
\end{proof}
\begin{defin}
	\textit{Гомотопией} между отображениями $f,g\colon X\to Y$ называется отображение $H\colon X\times[0,1]\to Y$ такое, что $H|_{t=0}=f$ и $H|_{t=1}=g$, про сами отображения будем говорить, что они \textit{гомотопны} и писать $H\colon f\simeq g$ или просто $f\simeq g$.
\end{defin}
\begin{prop}
	Отношение \textit{``быть гомотопным''} на пространстве $C(X,Y)$ --- отношение эквивалентности.
\end{prop}
\begin{proof}
	\textbf{Рефлексивность.} Отображение $f\in C(X,Y)$ гомотопно самому себе через гомотопию $H(x,t)=f(x)$.

	\textbf{Симметричность.} Если $H\colon f\simeq g$ для $f,g\in C(X,Y)$, то $\tilde{H}(x,t)=H(x,1-t)$ --- гомотопия между $g$ и $f$.

	\textbf{Транзитивность.} Пусть $F\colon f\simeq g$ и $G\colon g\simeq h$ для $f,g,h\in C(X,Y)$. Представим себе две копии цилиндра $X\times I$: с нижнего основания первого цилиндра бьёт $f$, с его верхнего основания и с нижнего основания второго цилиндра бьёт $g$, а $h$ определена на верхнем основании второго цилиндра. Чтобы определить гомотопию $H\colon f\simeq h$, ``склеим'' эти цилиндры по тем основаниям, на которых определено $g$ и сожмём полученный цилиндр в два раза. На нём можно задать отображение
		\begin{equation*}
			H(x,t)=\begin{cases}
				F(x,2t), 0\leqslant t<1/2,\\
				G(x,2t-1), 1/2\leqslant t\leqslant 1,
			\end{cases}
		\end{equation*}
	являющееся искомой гомотопией. Гомотопия $H$ непрерывна в силу леммы о склейке.
\end{proof}

Факторпространство $C(X,Y)/\simeq$ будем обозначать $[X,Y]$ --- это пространство классов гомотопных отображений из $X$ в $Y$. Гомотопический класс отображения $f\in C(X,Y)$ обозначим $[f]\in[X,Y]$. Топологические пространства вместе с пространствами классов гомотопных отображений образуют категорию $h\mathcal{T}op$.

\begin{prop}
	Пусть $f,g\colon X\to Y$, причём $f\simeq g$, а $f',g'\colon Y\to Z$, причём $f'\simeq g'$. Тогда $f'\circ f\simeq g'\circ g$.
\end{prop}

\begin{defin}
	Пространства $X$ и $Y$ \textit{гомотопически эквивалентны}, если существуют $f\colon X\to Y$ и $g\colon Y\to X$ такие, что $f\circ g=\id_Y$ и $g\circ f=\id_X$. Отображения $f$ и $g$ называют \textit{гомотопическими эквивалентностями}.
\end{defin}
Легко видеть, что это определение изоморфизма в $h\mathcal{T}op$.

Простейший пример --- $\RR^n\setminus\{0\}\simeq S^{n-1}$. Построим гомотопию между вложением сферы $i\colon S^{n-1}\incl{}\RR^n\setminus\{0\}$ и $p\colon\RR^n\to S^{n-1}$, $x\to x/||x||$, положим
	\begin{equation*}
		H(x,t)=tx+(1-t)x/||x||.
	\end{equation*}
\begin{defin}
	Пространство $X$ стягиваемо, если оно гомотопически эквивалентно точке $pt$.
\end{defin}
Примеры стягиваемых пространств: $I$, $\RR^n$, $\RR^{\infty}$, $S^{\infty}$.

По определению для доказательства стягиваемости нужно привести отображения $f\colon X\to pt$ и $g\colon pt\to X$ такие, что $f\circ g\colon pt\to pt\simeq\id_{pt}$ и $g\circ f\colon X\to X\simeq\id_X$. Отображение $X\to pt$ единственно, а первая гомотопность тривиально выполнена, поэтому нужно убедиться в том, что постоянное отображение $X\to X$ гомотопно тождественному.

Покажем, что $\RR^n$ стягиваемо. Пусть $f(x)=0$. Тогда искомая гомотопия --- это просто $H(x,t)=tx$.

\subsubsection{Свойство продолжения гомотопии.} Обсудим то, что в англоязычной литературе называется homotopy extension property (HEP).
\begin{defin}
	Отображение $r\colon X\to X$ --- \textit{ретракция}, если $r^2=r$.
\end{defin}
Ретракция --- это топологический аналог проектора. Эквивалентно, $r$ --- ретракция, если для $A=r(X)$ верно $r|_A=\id_A$. Множество $A$ называется \textit{ретрактом}. Если ретракция $r\colon X\to X$ гомотопна $\id_X$, то она называется \textit{деформационным ретрактом}.

Вместе с пространствами с отмеченными точками будем также рассматривать \textit{пары} $(X, A)$, где $A\subset X$. \textit{Отображение пар} $f\colon (X,A)\to (Y,A)$ --- это такое отображение, что $f(A)\subset B$. Ясно, что пары пространств вместе с отображениями пар образуют категорию, и что пространства с отмеченными точками --- их частный случай.

Будем говорить, что пара $(X,A)$ удовлетворяет \textit{свойству продолжения гомотопии} (или \textit{является парой Борсука}), если для любого отображения $f\colon X\to Y$ и гомотопии $F\colon A\times I\to Y$ таких, что $F|_{A\times\{0\}}=f|_A$, существует гомотопия $\hat{F}\colon X\times I\to Y$ такая, что $\hat{F}|_{X\times\{0\}}=f$ и $\hat{F}|_{A\times I}=F$. Иными словами, $(X,A)$ обладает HEP, если отображение $X\times\{0\}\cup A\times I\to Y$ можно продолжить до отображения $X\times I\to Y$. Вложение $A\incl{} X$, если  $(X,A)$ --- пара Борсука, называют \textit{корасслоением}.

Пусть $X$ --- это основание стакана, а $A$ --- граница этого основания, тогда $(X,A)$ --- пара Борсука, если отображение из всего стакана можно продолжить на весь цилиндр.

\subsection{Клеточные пространства и теорема о клеточной аппроксимации}
\begin{prop}
	Пусть $X$ --- клеточное пространство, а клеточное подпространство $A\subset X$ стягиваемо. Тогда $X\simeq X/A$.
\end{prop}