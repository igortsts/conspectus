\section{Выпуклость}
Явления выпуклости служат нулевым приближением в задачах оптимизации, потому что минимум выпуклой функции, заданной на выпуклом множестве, всегда существует и единственнен.

Выпуклую геометрию пронизывают идеи двойственности, я постараюсь явно указывать на симметрию везде, где это возможно.

Далее $X$ обозначает произвольное линейное пространство над $\RR$. Действие линейного функционала $p\in X^*$ будем обозначать через $\langle p,x\rangle$.

Пространство $\RR^n$ рассматривается вместе со скалярным произведением
	\begin{equation}
		\langle x,y\rangle =\sum_{i=1}^n x_iy_i,
	\end{equation}
где $x_i$, $y_i$, $i=\overline{1,n}$ --- компоненты векторов в стандартном базисе в $\RR^n$.

\subsection{Выпуклые множества и функции}
\begin{defin}
	Пусть $A,B\subset X$, $t\in\RR$. Определим
		\begin{eqnarray}
			A+B&=&\{x+y|x\in X, y\in B\},\\
			tA&=&\{tx|x\in X\}.
		\end{eqnarray}
\end{defin}
\begin{defin}
	Множество $A$ центрально-симметрично, если $(-A)=A$.
\end{defin}
\begin{defin}
	Множество $A\subset X$ выпукло, если для всякой пары точек $x,y\in A$ и $\lambda\in(0,1)$
	\begin{equation}
		\lambda x+(1-\lambda)y\in A.
	\end{equation}
\end{defin}
Эквивалентно, $A$ выпукло, если $\lambda A+(1-\lambda) A\subset A$. Геометрически, множество $A$ выпукло, если со всякой парой точек, входящих в $A$, оно содержит и отрезок, их соединяющий.

Если $X$ --- топологическое векторное пространство, то \textit{телом} будем называть любой компакт с непустой внутренностью. Мы будем изучать выпуклые тела в $X$.

Самый важный пример выпуклого множества: если $||\cdot||$ --- норма на линейном пространстве $X$, то единичный шар $B^{||\cdot||}_1(0)\subset X$ в этой норме выпуклый. В самом деле, для $x,y\in X$ таких, что $||x||,||y||\leqslant 1$, и $\lambda\in(0,1)$
		\begin{equation*}
			||\lambda x+(1-\lambda)y||\leqslant |\lambda|||x||+|1-\lambda|||y||\leqslant 1.
		\end{equation*}
Простейшие свойства:
\begin{enumerate}
	\item если $A,B\subset X$ выпуклы, то $A+B$ и $tA$ выпуклы при произвольных $t\in\RR$.
	\item если $A\subset X$ и $B\subset Y$ выпуклы, то $A\times B\subset X\times Y$ выпукло.
	\item если $A_{\alpha}\subset X$ выпуклы при всех $\alpha\in\mathcal{A}$, то $\bigcap_{\alpha\in\mathcal{A}} A_{\alpha}$ выпукло.
\end{enumerate}
\begin{defin}
	Пусть подмножество $A\subset X$ выпуклое. Функция $f\colon A\to\RR$ выпуклая, если её надграфик $\epi(f)=\{(x,y)\in A\times\RR| \ x\in A,\ y\geqslant f(x)\}$ --- выпуклое множество.
\end{defin}

\subsection{Неравенства}
\textbf{Неравенство Брунна-Минковского}
\begin{theorem}
Пусть $A,B$ --- положительно определённые матрицы размера $n\times n$. Тогда
	\begin{equation}
		(\det(A+B))^\{1/n\}\geqslant(\det A)^{1/n}+(\det B)^{1/n},
	\end{equation}
причём равенство достигается тогда и только тогда, когда $B=cA$ для какого-то $c\in\RR$.
\end{theorem}
\begin{proof}
	Воспользуемся положительной определённостью, тогда найдётся матрица $P$ такая, что $A=P^2$. Тогда
	\begin{eqnarray*}
		(\det(P^2+B))^{1/n}&=&(\det P^2\det(I+(P^{-1}BP^{-1})))^{1/n}\\
						   &=&(\det P)^{2/n}(\det(I+P^{-1}BP^{-1}))^{1/n}.
	\end{eqnarray*}
	С другой стороны,
	\begin{eqnarray*}
		(\det P)^{2/n}+(\det B)^{1/n}=(\det P)^{2/n}(1+(\det P^{-1}BP^{-1})^{1/n}).
	\end{eqnarray*}
	Обозначив $C=P^{-1}BP^{-1}$, получаем, что нужное неравенство эквивалентно
	\begin{equation}\label{eq:1}
		(\det(I+C))^{1\n}\geqslant 1+(\det C)^{1/n}.
	\end{equation}
	Пусть $\{\lambda_1,\ldots,\lambda_n\}$ -- спектр оператора $C$. Тогда утверждение теоремы эквивалентно неравенству
	\begin{equation}
		(1+\lambda_1)^{1/n}\cdot\ldots\cdot(1+\lambda_n)^{1/n}\geqslant 1+(\lambda_1\cdot\ldots\cdot\lambda_n)^{1/n}
	\end{equation}
\end{proof}

\subsection{Функционал Минковского и норма}
Нормы на $\RR^n$ двойственны центрально-симметричным выпуклым телам в $\RR^n$: каждой норме можно поставить в соответствие единичный шар в этой норме, каждому центрально-симметричному выпуклому телу можно поставить в соответствие его функционал Минковского, являющийся нормой.
\begin{defin}
	Пусть $A\subset X$. Определим \textit{функционал Минковского} $\mu_A\colon X\to\RR$ как
		\begin{equation}
			\mu_A(x)=\inf\{t>0|x\in tA\}.
		\end{equation}
\end{defin}
\begin{prop}\label{prop:mink}
	Функционал Минковского $\mu_A(\cdot)$ произвольного множества $A\subset\RR^n$ положительно однородный, то есть $\mu_A(\lambda x)=\lambda\mu_A(x)$ для $\lambda>0$. Если $A$ ограничено и $0\in\int A$, то $\mu_A(x)>0$ для $x\neq 0$ и $\mu_A(0)=0$. Если $A$ выпукло, то выполняется неравенство треугольника: $\mu_A(x+y)\leqslant\mu_A(x)+\mu_A(y)$.
\end{prop}
\begin{prop}
	Пусть $A\subset\RR^n$ --- выпуклое центрально-симмметричное тело. Тогда $\mu_A(\cdot)$ --- норма на $\RR^n$.
\end{prop}
\begin{proof}
	Элементарное следствие предложения~\ref{prop:mink}.
\end{proof}

\subsection{Преобразование Лежандра}
\begin{defin}
	Пусть $f\colon X\to\RR$ --- некоторая функция. \textit{Преобразованием Лежандра} функции $f$ называют функцию $f^*\colon X\to\RR$, определённую в каждой точке как
		\begin{equation}
			f^*(y)=\sup_{x\in X}(\langle x,y\rangle-f(x)).
		\end{equation}
\end{defin}

\subsection{Опорная функция}
\begin{defin}
	Пусть $A\subset X$. \textit{Опорной функцией} множества $A$ называется функция $s_A\colon X^*\to\RR$, определённую в каждой точке как
		\begin{equation}
			s_A(p)=\sup\{\langle p,x\rangle|\ x\in A\}.
		\end{equation}
\end{defin}

\subsection{Двойственность Минковского}
\begin{defin}
	Пусть $A\subset\RR^n$. \textit{Полярой} множества $A$ называется множество
		\begin{equation}
			A^{\circ}=\{y\in\RR^n|\forall x\in A\ \langle x,y\rangle\leqslant 1\}.
		\end{equation}
\end{defin}
\begin{prop}
	Пусть $K\subset\RR^n$ --- выпуклое тело, содержащее 0 в своей внутренности. Тогда
		\begin{equation}
			\mu_A(x)=s_{A^{\circ}}(x).
		\end{equation}
\end{prop}

\subsection{Расстояние Банаха-Мазура и расстояние Хаусдорфа}
%как из одного расстояния можно получить другое?