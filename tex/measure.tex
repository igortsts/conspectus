\section{Мера и интеграл}

{\bf Чтобы научиться интегрировать функции, значения которых можно складывать друг с другом и умножать на числа, нужно уметь измерять подмножества их области определения. Это ясно из интуитивного представления об интеграле как о сумме
\begin{equation*}
	\int_X f(x)d\mu(x)=\sum_{x\in X} f(x)\mu(f^{-1}(x)).
\end{equation*}}

\subsection{Алгебры, кольца и полукольца}
Пусть $X$ --- множество. 
\begin{defin}
	Семейство подмножеств $\mathcal{A}\subset 2^X$ будем называть \textit{алгеброй подмножеств множества $X$} или \textit{алгеброй на $X$}, если
		\begin{enumerate}
			\item $\varnothing\in\mathcal{A}$;
			\item $A\in\mathcal{A}\Rightarrow\overline{A}\in\mathcal{A}$;
			\item $A,B\in\mathcal{A}\Rightarrow A\cup B\in\mathcal{A}$.
		\end{enumerate}
\end{defin}
Как алгебраическая структура алгебра множеств --- это ассоциативная алгебра с единицей, где роль единицы играет всё $X$, умножения --- пересечение множеств, а сложения --- симметрическая разность $A\triangle B=(A\setminus B)\cup(B\setminus A)$.

Вследствие замкнутости относительно операций дополнения и объединения алгебра также замкнута относительно операции пересечения множеств, что элементарно следует из правил де Моргана: $A\cap B=\overline{\overline{A}\cup\overline{B}}$. Из индукционнных соображений тривиально следует замкнутость относительно конечных объединений. Из этого, однако, не следует замкнутость относительно счётных объединений.
\begin{defin}
	Если в предыдущем определении требование (3) заменить на
		\begin{enumerate}
			\item $\{A_i\}_{i\in\NN}\subset\mathcal{A}\Rightarrow \bigcup_{i\in\NN} A_i\in\mathcal{A}$,
		\end{enumerate}
	то $\mathcal{A}$ --- \textit{$\sigma$-алгебра подмножеств множества $X$}.
\end{defin}

Понятие алгебры достаточно для определения измеримого пространства.
\begin{defin}
	Пару $(X,\mathcal{A})$, где $\mathcal{A}$ --- алгебра подмножеств $X$, будем называть \textit{измеримым пространством}. Множества $A\in\mathcal{A}$ назовём \textit{$\mathcal{A}$-измеримыми} или просто \textit{измеримыми}, если ясно, какая на $X$ определена алгебра.
\end{defin}
\begin{defin}
	Пусть $(X,\mathcal{A})$ --- измеримое пространство, $Y\subset X$. Тогда на $Y$ индуцируется структура измеримого пространства с алгеброй множеств $\mathcal{B}=\{Y\cap A|A\in\mathcal{A}\}$.
\end{defin}

Обычно для определения меры на конкретном измеримом пространстве её не задают на всей алгебре подмножеств $X$, а на более бедном, но обозримом семействе подмножеств множества $X$, например, кольце или полукольце множеств.
\begin{defin}
	Семейство подмножеств $\mathcal{R}\subset 2^X$ --- \textit{кольцо подмножеств множества $X$} или \textit{кольцо на $X$}, если
	\begin{enumerate}
		\item $A,B\in\mathcal{R}\Rightarrow A\cap B\in\mathcal{R}$;
		\item $A,B\in\mathcal{R}\Rightarrow A\triangle B\in\mathcal{R}$.
	\end{enumerate}
\end{defin}
Отметим, что $A\setminus B=A\triangle(A\cap B)$, откуда следует, что всякая алгебра является кольцом. Обратное верно тогда и только тогда, когда $X\in\mathcal{A}$.

\begin{prop}
	Пусть $\{\mathcal{R}_{\alpha}\}_{\alpha\in I}$ --- набор колец подмножеств множества $X$. Тогда семейство подмножеств $\mathcal{R}=\bigcap\limits_{\alpha\in I} R_{\alpha}$ само является кольцом подмножеств множества $X$.
\end{prop}
Это утверждение обеспечивает существование наименьшего кольца подмножеств множества $X$, содержащее данное семейство подмножеств $\mathcal{M}\subset 2^X$.
\begin{defin}
	Кольцо множеств $\mathcal{R}(\mathcal{M})=\bigcap\limits_{\mathcal{M}\subset\mathcal{F}\subset 2^X} \mathcal{F}$, где $\mathcal{F}$ --- кольцо подмножеств множества $X$, будем называть \textit{кольцом подмножеств множества $X$, порождённым семейством $Y$}. 
\end{defin}
Из определения ясно, что данное семейство $\mathcal{F}$ подмножеств множества $X$ порождает единственное кольцо, которое его содержит.
\begin{defin}
	Пусть $(X,\tau)$ --- топологическое пространство. \textit{Борелевской $\sigma$-алгеброй} на $X$ называется алгебра $\mathcal{A}(\tau)$, порождённая открытыми подмножествами пространства $X$.
\end{defin}

Полукольцо удовлетворяет более слабому условию, чем (2) в определении кольца.
\begin{defin}
	Семейство подмножеств $\mathcal{H}\subset 2^X$ --- \textit{полукольцо подмножеств множества $X$}, если
	\begin{enumerate}
		\item $A,B\in\mathcal{H}\Rightarrow A\cap B\in\mathcal{H}$;
		\item $A,B\in\mathcal{H}\Rightarrow\exists\ A_1,\ldots,A_n\in\mathcal{H}\such \forall\ i\neq j\ A_i\cap A_j=\varnothing,\ A\setminus B=A_1\cup\ldots\cup A_n$;
	\end{enumerate}
\end{defin}
Всякое кольцо, очевидно, является полукольцом.

Приведём примеры семейств множеств.
\begin{enumerate}
	\item Семейства $\{\varnothing, X\}$ и $2^X$ являются $\sigma$-алгебрами на $X$.
	\item Пусть $X=\NN$. Семейство конечных подмножеств множества $X$ образует кольцо на $X$, но не алгебру.
	\item Пусть $X=\RR$. Конечные объединения промежутков вида $(a,b)$, $(a,b]$, $[a,b)$ и $[a,b]$ образуют алгебру.
	\item Множество $P_{a,b}\subset\RR^n$ будем называть \textit{блоком}, если найдутся векторы $a=(a_1,\ldots,a_n),b=(b_1,\ldots,b_n)\in\RR^n$ такие, что $a_i\leqslant b_i$, $i\in\{1,\ldots,n\}$, и
		\begin{equation*}
			\prod_{i=1}^n (a_i,b_i)\subset P_{a,b}\subset\prod_{i=1}^n [a_i,b_i],
		\end{equation*}
		и \textit{блочным}, если оно представимо в виде конечного объединения непересекающихся блоков. Семейство блочных подмножеств пространства $\RR^n$ образует полукольцо\label{itm:block}.
\end{enumerate}

\subsection{Мера}
\begin{defin}
	Пусть $\mathcal{S}$ --- некоторое семейство подмножеств множества $X$. Функцию $f\colon\mathcal{S}\to\RR$ такую, что $f(A\cup B)=f(A)+f(B)$ для всяких неперескающихся $A,B\in\mathcal{S}$ таких, что $A\cup B\in\mathcal{S}$, назовём \textit{конечно-аддитивной}, или просто \textit{аддитивной}.
\end{defin}
Очевидно, что для конечно-аддитивного отображения $f\colon\mathcal{S}\to\RR$ и конечного набора непересекающихся множеств $\{A_i\}_1^n$ выполнено $f\left(\bigcup A_i\right)=\sum f(A_i)$, что, конечно, в общем случае неверно для счётного набора. Если это равенство имеет место для счётного набора непересекающихся множеств, функция $f$ называется \textit{счётно-аддитивной}.
\begin{defin}
	Пусть $\mathcal{R}$ --- кольцо ($\sigma$-кольцо) на $X$. \textit{Конечно-аддитивная мера} \textit{(счётно-аддитивная)} на множестве $X$ --- это конечно-аддитивная функция $\mu\colon\mathcal{R}\to(-\infty,+\infty]$ такая, что $\mu(\varnothing)=0$.
\end{defin}
\begin{defin}
	Тройка $(X,\mathcal{A},\mu)$, где $\mathcal{A}$ --- $\sigma$-алгебра, а $\mu$ --- счётно-аддитивная мера, заданная на ней, называется \textit{пространством с мерой}.
\end{defin}
Приведём некоторые примеры.
\begin{enumerate}
	\item Рассмотрим кольцо конечных подмножеств множества $X$. Определим на нём меру как
		\begin{equation*}
			\mu(A)=\begin{cases}
					|A|, & \mbox{если } A\mbox{ конечно} \\
					+\infty, & \mbox{если } A\mbox{ бесконечно}.
				\end{cases}
		\end{equation*}
		Эта мера называется \textit{считающей}. Она конечно-аддитивна, но не счётно-аддитивна.
	\item Выберем точку $x\in X$. Определим на $X$ меру как
		\begin{equation*}
			\mu_x(A)=\begin{cases}
					1, & \mbox{если } x\in A\\
					0, & \mbox{если } x\not\in A.
				\end{cases}
		\end{equation*}
		Эта мера называется \textit{дираковской}. Она задана на $\sigma$-алгебре всех подмножеств множества $X$ и счётно-аддитивна.
	\item Продолжая пример~\ref{itm:block}, определим на блоке $P_{a,b}\subset\RR^n$ меру как
		\begin{equation*}
			\mu_0(P_{a,b})=\prod_{i=1}^n (b_i-a_i).
		\end{equation*}
		Для блочного множества $B=\bigcup\limits_{j=1}^m P_j$, составленного из непересекающихся блоков $P_j\subset\RR^m$, положим
		\begin{equation*}
			\mu_0(B)=\sum_{j=1}^m\mu_0(P_j).
		\end{equation*}
		Это конечно-аддитивная функция, заданная на полукольце блочных множеств.\\
		\TODO{дописать}
\end{enumerate}
\begin{defin}
	Пусть $(X,\mathcal{A})$, $(Y,\mathcal{B})$ --- измеримые пространства. Будем говорить, что отображение $f\colon(X,\mathcal{A})\to(Y,\mathcal{B})$ \textit{$\mathcal{A}/\mathcal{B}$-измеримое}, если для всякого $\mathcal{B}$-измеримого множества $B\subset Y$ его прообраз $f^{-1}(B)\subset X$ $\mathcal{A}$-измерим. 
\end{defin}
Смысл определения заключается в том, что если $(X,\mathcal{A},\mu)$ --- пространство с мерой, то измеримое отображение $f\colon(X,\mathcal{A},\mu)\to(Y,\mathcal{B})$ индуцирует меру $\tilde{\mu}$ на $Y$, определяемую на измеримых множествах $B\subset Y$ как $\tilde{\mu}(B)=\mu(f^{-1}(B))$.
	\begin{equation*}
		\begin{tikzcd}
			\mathcal{A} \arrow{r}{\mu} \arrow{d}{f} & \RR \\
			\mathcal{B} \arrow{ru}{\tilde{\mu}} & 
		\end{tikzcd}
	\end{equation*}
Измеримые пространства в качестве объектов вместе с измеримыми отображениями в качестве морфизмов образуют \textit{категорию измеримых пространств} $\mathcal{M}eas$.

\subsection{Продолжение меры}
\TODO{Теорема Каратеодори}

\subsection{Мера и вероятность}

\subsection{Вариация}

\subsection{Произведение мер}
\TODO{категорный смысл}
Пусть $(X,\mathcal{A})$ и $(Y,\mathcal{B})$ --- измеримые пространства. Прямое произведение $X\times Y$ тогда естественно наделяется структурой измеримого пространства, если определить прямое произведение алгебр $\mathcal{A}\otimes\mathcal{B}=\{A\times B|A\in\mathcal{A},B\in\mathcal{B}\}$.

\subsection{Мера Лебега}

\subsection{Интеграл}
Следующая конструкция интеграла числовой функции, видимо, является наиболее общей.
\begin{defin}
	Функция $f\colon X\to Y$ \textit{простая}, если множество множество $f(X)$ конечно.
\end{defin}
% \begin{defin}
% 	Разбиение $\{A_i\}_1^n$ пространства с мерой $(X,\mathcal{A},\mu)$ называется \textit{допустимым для простой функции} $f\colon X\to Y$, если для любого $y\in Y$ $f^{-1}(y)$  
% \end{defin}
Пусть $(X,\mathcal{A},\mu)$ --- пространство с мерой. Рассмотрим неотрицательную простую функцию $f\colon X\to\RR$, принимающую на измеримых множествах $\{A_i\}_1^n$, образующих разбиение $X$, значения $c_i=f(A_i)$.
\begin{defin}
	\textit{Интегралом неотрицательной простой функции} $f\colon X\to\RR$ назовём сумму
	\begin{equation}
		\int_X fd\mu=\sum_{i=1}^n c_i\mu(A_i).
	\end{equation}
\end{defin}
Корректность определения, а именно независимость от разбиения пространства $X$, необходимо проверить
\begin{prop}
	
\end{prop}
\begin{defin}
	Пусть
\end{defin}

\subsection{Изотропические меры}