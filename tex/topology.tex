\section{Общая топология}
Топология на множестве --- необходимая структура для определения непрерывных отображений. Всюду в этой главе под пространством мы понимаем топологическое пространство, а под отображениями --- непрерывные отображения между ними.

\subsection{Топологическое пространство и непрерывные отображения}
\begin{defin}\label{defin:topol}
	Пусть $X$ --- множество. Топологией на $X$ назовём семейство $\tau$ подмножеств множества $X$, удовлетворяющее следующим требованиям:
	\begin{enumerate}
		\item $\emptyset\in\tau$, $X\in\tau$;
		\item для всякого конечного набора подмножеств $\{X_i\}_{i=1}^n\subset\tau$ верно $\bigcap\limits_{i=1}^n X_i\in\tau$;
		\item для всякого набора множеств $\{X_{\alpha}\}_{\alpha\in A}\subset\tau$ верно $\bigcup\limits_{\alpha\in A} X_{\alpha}\in\tau$.
	\end{enumerate}
\end{defin}
\begin{defin}
	Пусть $\tau$ -- топология на $X$. Если $U\in\tau$, то $U$ называют \textit{открытым} подмножеством $X$ (в этой топологи). Множество $X$ замкнуто, если его дополнение открыто.
\end{defin}
Поясним то, что написано. С первым пунктом определения~\ref{defin:topol} всё должно быть более-менее ясно: пустое множество и всё множество будем считать открытыми. Согласно второму пункту, \textit{пересечение} всякого \textit{конечного} семейства открытых множеств снова открыто. Третий же пункт утверждает, что \textit{объединение} \textit{любого} семейства открытых множеств открыто.

Вспомним школьное определение непрерывности в точке.
\begin{defin}
	Функция $f\colon\RR\to\RR$ непрерывна в точке $x_0\in\RR$, если $\displaystyle\lim_{x\to x_0} f(x)=f(x_0)$.
\end{defin}
Посмотрим, как это определение обобщить. Для начала, его можно развернуть, используя $\varepsilon-\delta$-формализм:
\begin{equation}
	\forall\varepsilon>0\ \exists\delta\such\forall x\such |x-x_0|<\delta\ |f(x)-f(x_0)|<\varepsilon.
\end{equation}
Первый возможный шаг --- это замена модулей разности на \textit{расстояния} между точками. Напомним, что \textit{метрикой} или расстоянием на множестве $X$ называют функцию $\rho\colon X\times X\to\RR$, удовлетворяющую аксиомам:
\begin{enumerate}
	\item $\forall x,y\in X\ \rho(x,y)\leqslant 0$;
	\item $\forall x,y\in X\ \rho(x,y)=0\Leftrightarrow x=y$;
	\item $\forall x,y\in X\ \rho(x,y)=\rho(y,x)$;
	\item $\forall x,y,z\in X\ \rho(x,z)\leqslant\rho(x,y)+\rho(y,z)$.
\end{enumerate}
Множество $X$ с введёной на ней метрикой $\rho$ называют \textit{метрическим пространством} и пишут $(X,\rho)$.\\
Модуль разности $d(x,y)=|x-y|$ удовлетворяет всем четырём аксиомам. В сущности, при доказательстве свойств предела ничем, кроме этих свойств модуля, мы не пользуемся. Значит, о конкретном виде метрики можно не думать, нужны лишь её свойства. Теперь мы можем перенести определение непрерывности в точке с числовых функций на отображения между множествами, на которых введена метрика.
\begin{defin}
	Пусть $(X,\rho)$, $(Y,d)$ --- метрические пространства. Функция $f\colon X\to Y$ непрерывна в точке $x_0\in X$, если
	\begin{equation}
		\forall\varepsilon>0\ \exists\delta\such\forall x\such \rho(x,x_0)<\delta\ d(f(x),f(x_0))<\varepsilon.
	\end{equation}
\end{defin}
Можно пойти дальше. В метрическом пространстве $(X,\rho)$ определим \textit{$\delta$-окрестность точки $x$} как множество $U_{\delta}(x)=\{y\in X\ |\ \rho(x,y)<\delta\}$. Определение непрерывности тогда можно записать как
	\begin{equation}
		\forall\varepsilon>0\ \exists\delta\such\forall x\in U_{\delta}(x_0)\ f(x)\in U_{\varepsilon}(f(x_0)),
	\end{equation}
или, ещё короче,
	\begin{equation}
		\forall\varepsilon>0\ \exists\delta\such f(U_{\delta}(x_0))\subset U_{\varepsilon}(f(x_0)).
	\end{equation}
Повторим фокус снова: забудем про внутреннее устройство окрестности. Будем теперь считать, что каждой точке множества приписано семейство множеств, называемых окрестностями этой точки, свойства которых мы потом отдельно выделим. Максимально общо, определение непрерывности в точке теперь выглядит так, если мы предполагаем, что на множествах $X$ и $Y$ введены эти системы окрестностей.
\begin{defin}
	Функция $f\colon X\to Y$ непрерывна в точке $x_0\in X$, если
	\begin{equation}
		\forall U(f(x_0))\ \exists V(x_0)\such f(V(x_0))\subset U(f(x_0)).
	\end{equation}
\end{defin}

Приведём первые примеры.
\begin{enumerate}
	\item На всяком множестве $X$ можно рассматривать топологии $\{\varnothing, X\}$ и $2^X$. Последнюю также называют \textit{дискретной}: одноточечные множества открыты и замкнуты и каждую точку можно отделить от всех остальных окрестностью, состоящей из её одной.
	\item На множестве $\{0,1,2\}$ $\{\varnothing, \{1\}, \{0,1,2\}\}$ --- топология.
	\item На прямой $X=\RR$ открытыми объявляются все интервалы вида $(a,b)$, где $a,b\in\RR\cup\{\pm\infty\}$.
	\item На отрезке $I=[0,1]$
\end{enumerate}

Класс топологических пространств вместе с непрерывными отображениями в качестве морфизмов составляют \textit{топологическую категорию} $\mathcal{T}op$. Конечным объектом в этой категории будет одноточечное пространство $\{*\}$, так как в него существует единственное отображение. О суммах и произведениях в этой категории речь пойдёт в следующем параграфе. 

Топологии на множестве $X$ можно сравнивать: если $\tau_1$ и $\tau_2$ --- топологии на $X$, то говорят, что $\tau_1$ \textit{сильнее (тоньше)} $\tau_2$ и что $\tau_2$ \textit{слабее (грубее)} $\tau_1$, если $\tau_2\subset\tau_1$. Если никакое включение не выполняется, то топологии не сравнивают. Так, топология $\{\varnothing, X\}$ --- слабейшая топология на $X$, а $2^X$ --- сильнейшая.

В задании топологий на множестве важны понятия базы и предбазы.
\begin{defin}
	Семейство подмножеств $\beta\subset\tau$ пространства $X$ --- \textit{база топологии $\tau$ на $X$}, если всякое открытое множество представимо в виде (произвольного) объединения открытых множеств из $\beta$.
\end{defin}
\begin{defin}
	Семейство подмножеств $\sigma\subset\tau$ пространства $X$, конечные пересечения множеств которого образуют базу топологии $\tau$ на $X$, --- это \textit{предбаза} топологии $\tau$.
\end{defin}
Предбаза --- это набор множеств, ``порождающий'' топологию. Пусть имеется семейство $\sigma$ подмножеств множества $X$. Мы хотим, чтобы эти множества были открыты в некоторой топологии, причём желательно, чтобы она не содержала ``лишних'' открытых множеств. Тогда из этого набора нужно получить всевозможные конечные пересечения входящих в него множеств, а то, что получилось, любым образом прообъединять. Эквивалентно, рассмотрим множество $\mathcal{T}(\sigma)$ всех топологий на $X$, которые содержат в себе семейство $\sigma$. Тогда $\sigma$ --- предбаза топологии $\tau_{\sigma}=\bigcap\limits_{\tau\in\mathcal{T}(\sigma)}\tau$.

\begin{defin}
	Точка $x\in X$ --- \textit{точка прикосновения $X$}, если каждая её окрестность $U$ содержит ещё какую-то точку из $X$, то есть $(U\setminus{X})\cap X\neq\varnothing$.
\end{defin}
\begin{prop}
	Множество $A\subset X$ замкнуто тогда и только тогда, когда содержит все свои точки прикосновения.
\end{prop}
\begin{proof}

\end{proof}
Введения топологии достаточно, чтобы определить \textit{сходимость}.
\begin{defin}
	Пусть $\{x_n\}_{n=1}^{\infty}\subset X$. Тогда $x_i\rightarrow x$, если для всякой окрестности $U$ точки $x$ найдётся $n\in\NN$ такой, что для всех $m\geqslant n$ $x_m\in U$.
\end{defin}

\subsection{Компактность}
Без компактности в топологии абсурд и коррупция.
\begin{defin}
	Говорят, что семейство $\mathcal{U}=\{U_{\alpha}\}_{\alpha\in\mathcal{A}}\subset 2^X$ \textit{покрывает} (или является покрытием) множества $A\subset X$, если $A\subset\bigcup\limits_{\alpha\in\mathcal{A}} U_{\alpha}$. Если $\mathcal{U}'\subset\mathcal{U}$ и $\mathcal{U}'$ покрывает $A$, то $\mathcal{U}'$ --- подпокрытие покрытия $\mathcal{U}$.
\end{defin}
\begin{defin}
	Пусть $K\subset X$ таков, что из любого покрытия $\mathcal{U}=\{U_{\alpha}\}_{\alpha\in\mathcal{A}}\subset 2^X$, состоящего из открытых подмножеств пространства $X$, можно выделить конечное подпокрытие $\mathcal{U}'=\{U_{\alpha_i}\}_{i=1}^n$. Тогда пространство $K$ называют \textit{компактом}.
\end{defin}
\begin{prop}
	Пусть $X$ компактно, а $f\colon X\to Y$ непрерывно. Тогда $f(X)\subset Y$ компактно.
\end{prop}
\begin{proof}
	Пусть $\mathcal{U}=\{U_{\alpha}\}_{\alpha\in\mathcal{A}}$ --- открытое покрытие $f(X)$, тогда $\{f^{-1}(U_{\alpha})\}_{\alpha\in\mathcal{A}}$ --- открытое покрытие $X$. Выделим из него открытое подпокрытие $\{f^{-1}(U_{\alpha_i}\}_{i=1}^n$. Значит, $\{U_{\alpha_i}\}_{i=1}^n$ --- открытое подпокрытие покрытия $\mathcal{U}$.
\end{proof}

Важное для приложений свойство, вытекающее из компактности, --- \textit{секвенциальная компактность}. В матанализе мы любим изучать свойства последовательностей.
\begin{defin}
	Пространство $X$ \textit{секвенциально компактно}, если из всякой последовательности элементов из $X$ можно выделить сходящуюся подпоследовательность.
\end{defin}
\begin{prop}
	Компактное пространство секвенциально компактно.
\end{prop}

\subsection{Топологические конструкции}
\begin{defin}
	Пусть $f\colon X\to Y$ --- произвольное отображение, а на $Y$ определена топология. \textit{Начальная топология (initial topology) на $X$ относительно отображения $f$} --- это слабейшая топология, относительно которой отображение $f$ непрерывно.
\end{defin}
Несложно дать явное задание этой топологии, стоит только вспомнить определение непрерывности: необходимо и достаточно, чтобы были открыты все множества вида $f^{-1}(U)$, где $U$ открыто в $Y$, то есть $\sigma=\{f^{-1}(U)|U\text{ --- открыто в Y}\}$ --- предбаза начальной топологии. Эта терминология не может считаться устоявшейся в русской литературе и не встречается за пределами этого конспекта.

Двойственным понятием будет \textit{конечная топология}.
\begin{defin}
	Пусть $f\colon X\to Y$ --- произвольное отображение, а на $X$ определена топология. \textit{Конечная топология (final topology) на $Y$ относительно отображения $f$} --- это сильнейшая топология, относительно которой отображение $f$ непрерывно.
\end{defin}
Зададим явно и эту топологию: множество $U\subset Y$ открыто тогда и только тогда, когда его прообраз $f^{-1}(U)$ открыт в $X$.
\begin{quest}
	Что меняется, когда вместо одного отображения $f\colon X\to Y$ рассматривается семейства отображений $f_i\colon X\to Y_i$ и $f_i\colon X_i\to Y$ в определениях начальной и конечной топологии соответственно?
\end{quest}
Приведём классические примеры этих топологий.

\textbf{Индуцированная топология.} Пусть $A\subset X$. Рассмотрим \textit{отображение включения} $i\colon A\incl{} X$, $i(a)=a$. Если на $X$ есть топология, то, чтобы задать её на $A$, можно потребовать, чтобы все открытые в $A$ множества имели вид $U\cap A$, где $U$ открыто в $X$. Элементарно проверяется, что такие множества действительно образуют топологию на $A$ и что $i$ оказывается непрерывным. Более того, это слабейшая топология, относительно которой включение $i$ непрерывно.

\textbf{Фактортопология.} Пусть $X$ --- топологическое пространство, на котором определено \textit{отношение эквивалентности} $\sim$. Рассмотрим множество $X/\sim$ и каноническую \textit{проекцию} $\pi\colon X\to X/\sim$, $x\to[x]$. Множество $X/\sim$ можно снабдить топологией, потребовав, чтобы $U\subset X/\sim$ было открыто тогда и только тогда, когда $\pi^{-1}(U)$ открыт в $X$.

Полезный пример фактортопологии --- \textit{стягивание}. Пусть $A\subset X$, а точки $x,y\in X$ связаны отношением эквивалентности $\sim$ тогда и только тогда, когда $x,y\in A$. Тогда говорят, что $X/A$ получено стягиванием подпространства $A$ в точку.

\textbf{Топология произведения.} Пусть $X$, $Y$ --- топологические пространства. Прямое произведение $X\times Y$ можно наделить естественной топологией, потребовав, чтобы канонические проекции $\pi_X\colon X\times Y\to X$ и $\pi_Y\colon X\times Y\to Y$ были непрерывны. Аналогично топология определяется для произведения произвольного семейства пространств $\{X_{\alpha}\}_{\alpha\in\mathcal{A}}$, тогда требуется, чтобы все $\pi_{\beta}\colon\prod X_{\alpha}\to X_{\beta}$ были непрерывны. Она называется \textit{тихоновской}. В случае конечного произведения её задание тривиально: множества вида $U\times V$, где $U\subset X$ и $V\subset Y$ открыты, образуют базу в $X\times Y$.

Если на $X\times Y$ ввести тихоновскую топологию, это пространство станет категорным произведением пространств $X$ и $Y$.

Классический результат --- теорема Тихонова.
\begin{theorem}
	Если все $X_{\alpha}$, $\alpha\in\mathcal{A}$, компактны. Тогда $\prod\limits_{\alpha\in\mathcal{A}}$ $X_{\alpha}$ компактно.
\end{theorem}
\begin{proof}
Докажем теорему для произведения двух пространств, тогда то же будет верно для любого конечного произведения.

Утверждение теоремы в случае бесконечного числа множителей эквивалентно аксиоме выбора.
\end{proof}

\textbf{Несвязное объединение}. Теоретико-множественное несвязное объединение определяется следующим образом. Пусть $A$ и $B$ множества, возможно, имеющие ненулевое пересечение. Мы хотим, чтобы в несвязное объединение элементы из пересечения вошли ``дважды'': как элементы $A$ и как элементы $B$. Положим $A\sqcup B=\{(a,0), (b,1)|a\in A, b\in B\}$. Вместе с этим определяются вложения $i_A\colon A\incl{} A\sqcup B$, $a\mapsto (a,0)$. Эта конструкция является категорной суммой в категории множеств.

Если $X$ и $Y$ --- топологические пространства, то $A\sqcup B$ наделяется естественной топологией и оказывается суммой в топологической категории.

\textbf{Надстройка} Для пространства $X$ определим
	\begin{equation*}
		\Sigma X=X\times I
	\end{equation*}

\subsection{Пространства с отмеченной точкой}
Так будем называть пары $(X,x_0)$, где $X$ --- топологическое пространство, а $x_0\in X$. Про непрерывное отображение $f\colon(X,x_0)\to (Y,y_0)$ будем говорить, что оно \textit{сохраняет отмеченную точку}, если $f(x_0)=y_0$. Пространства с отмеченной точкой вместе с отображениями, их сохраняющими, в качестве морфизмов, образуют \textit{пунктированную категорию} $\mathcal{T}op^*$.

В терминах пространств с отмеченной точкой даётся определение фундаментальной группы, высших гомотопических групп и групп гомологий.

Для таких пространств определим \textit{букет}, являющийся суммой в категории $\mathcal{T}op^*$.
\begin{defin}
	Пусть $\{(X_{\alpha},x_{\alpha}\}_{\alpha\in\mathcal{A}}$ --- семейство пространств с отмеченными точками. Их \textit{букетом} назовём пространство
		\begin{equation*}
			\bigvee\limits_{\alpha\in\mathcal{A}}X_{\alpha}=\bigsqcup\limits_{\alpha\in\mathcal{A}}X_{\alpha}\big/\{x_\alpha\}_{\alpha\in\mathcal{A}}.
		\end{equation*}
\end{defin}
Иными словами, все пространства приклеили друг к другу по отмеченным точкам: так, букет двух окружностей --- восьмёрка.

\subsection{Гомотопия и гомотопическая эквивалентность}
\begin{defin}
	\textit{Гомотопией} между отображениями $f,g\colon X\to Y$ называется отображение $H\colon X\times[0,1]\to Y$ такое, что $H|_{t=0}=f$ и $H|_{t=1}=g$, про сами отображения будем говорить, что они \textit{гомотопны} и писать $f\simeq g$.
\end{defin}
\begin{prop}
	Отношение \textit{``быть гомотопным''} на пространстве $C(X,Y)$ --- отношение эквивалентности.
\end{prop}
Если попытаться придать точный смысл \textit{деформации}, то получится гомотопия. Нарисуем в воображении некоторый объект, будем считать, что он был получен как непрерывный образ $f(X)$ какого-то множества $X$ в $Y$. ``Раскадрируем'', как объект меняется с течением времени, получим серию картинок $f_0(X)=f(X), f_1(X),\ldots, f_n(X)\subset Y$. Эту серию можно собрать в одно отображение $H\colon X\times\{0,1,\ldots,n\}\to Y$. Пожелав, чтобы кадры менялись непрерывно, мы определим непрерывное отображение $H\colon X\times I\to Y$.

Факторпространство $C(X,Y)/\simeq$ будем обозначать $[X,Y]$ --- это пространство классов гомотопных отображений из $X$ в $Y$. Топологические пространства вместе с пространствами классов гомотопных отображений образуют категорию $h\mathcal{T}op$.

\begin{defin}
	Пространства $X$ и $Y$ \textit{гомотопически эквивалентны}, если существуют $f\colon X\to Y$ и $g\colon Y\to X$ такие, что $f\circ g=\id_Y$ и $g\circ f=\id_X$.
\end{defin}
Легко видеть, что это определение изоморфизма в $h\mathcal{T}op$.

\begin{defin}
	Пространство $X$ стягиваемо, если оно гомотопически эквивалентно точке $pt$.
\end{defin}

\subsection{Клеточные пространства и теорема о клеточной аппроксимации}

\subsection{Фундаментальная группа}
\TODO{теорема Брауэра о неподвижной точке, теорема Борсука-Улама}

\subsection{Теорема Зейферта-ван Кампена}
\TODO{свободное произведение}

\TODO{категорный смысл --- сохранение пушаутов}

\subsection{Фундаментальная группа клеточного пространства. Классификация двумерных поверхностей}

\subsection{Накрытия}
\TODO{свойство поднятия пути и гомотопии, универсальное накрытие, классификация накрытий, теорема Нильсена-Шраера}