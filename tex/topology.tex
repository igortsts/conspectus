\section{Общая топология}
Топология на множестве --- необходимая структура для определения непрерывных отображений. Всюду в этой главе под отображениями мы понимаем непрерывные отображения между непрерывными пространствами.

\subsection{Топологическое пространство и непрерывные отображения}
\begin{defin}\label{defin:topol}
	Пусть $X$ --- множество. Топологией на $X$ назовём семейство $\tau$ подмножеств множества $X$, удовлетворяющее следующим требованиям:
	\begin{enumerate}
		\item $\emptyset\in\tau$, $X\in\tau$;
		\item для всякого конечного набора подмножеств $\{X_i\}_{i=1}^n\subset\tau$ верно $\displaystyle\cap_{i=1}^n X_i\in\tau$;
		\item для всякого набора множеств $\{X_{\alpha}\}_{\alpha\in A}\subset\tau$ верно $\displaystyle\cup_{\alpha\in A} X_{\alpha}\in\tau$.
	\end{enumerate}
\end{defin}
\begin{defin}
	Пусть $\tau$ -- топология на $X$. Если $U\in\tau$, то $U$ называют открытым подмножеством $X$ (в этой топологи).
\end{defin}
Поясним то, что написано. С первым пунктом определения~\ref{defin:topol} всё должно быть более-менее ясно: пустое множество и всё множество будем считать открытыми. Согласно второму пункту, \textit{пересечение} всякого \textit{конечного} семейства открытых множеств снова открыто. Третий же пункт утверждает, что \textit{объединение} \textit{любого} семейства открытых множеств открыто.

Вспомним школьное определение непрерывности в точке.
\begin{defin}
	Функция $f\colon\RR\to\RR$ непрерывна в точке $x_0\in\RR$, если $\displaystyle\lim_{x\to x_0} f(x)=f(x_0)$.
\end{defin}
Посмотрим, как это определение обобщить. Для начала, его можно развернуть, используя $\varepsilon-\delta$-формализм:
\begin{equation}
	\forall\varepsilon>0\ \exists\delta\such\forall x\such |x-x_0|<\delta\ |f(x)-f(x_0)|<\varepsilon.
\end{equation}
Первый возможный шаг --- это замена модулей разности на \textit{расстояния} между точками. Напомним, что \textit{метрикой} или расстоянием на множестве $X$ называют функцию $\rho\colon X\times X\to\RR$, удовлетворяющую аксиомам:
\begin{enumerate}
	\item $\forall x,y\in X\ \rho(x,y)\leqslant 0$;
	\item $\forall x,y\in X\ \rho(x,y)=0\Leftrightarrow x=y$;
	\item $\forall x,y\in X\ \rho(x,y)=\rho(y,x)$;
	\item $\forall x,y,z\in X\ \rho(x,z)\leqslant\rho(x,y)+\rho(y,z)$.
\end{enumerate}
Множество $X$ с введёной на ней метрикой $\rho$ называют \textit{метрическим пространством} и пишут $(X,\rho)$.\\
Модуль разности $d(x,y)=|x-y|$ удовлетворяет всем четырём аксиомам. В сущности, при доказательстве свойств предела ничем, кроме этих свойств модуля, мы не пользуемся. Значит, о конкретном виде метрики можно не думать, нужны лишь её свойства. Теперь мы можем перенести определение непрерывности в точке с числовых функций на отображения между множествами, на которых введена метрика.
\begin{defin}
	Пусть $(X,\rho)$, $(Y,d)$ --- метрические пространства. Функция $f\colon X\to Y$ непрерывна в точке $x_0\in X$, если
	\begin{equation}
		\forall\varepsilon>0\ \exists\delta\such\forall x\such \rho(x,x_0)<\delta\ d(f(x),f(x_0))<\varepsilon.
	\end{equation}
\end{defin}
Можно пойти дальше. В метрическом пространстве $(X,\rho)$ определим \textit{$\delta$-окрестность точки $x$} как множество $U_{\delta}(x)=\{y\in X\ |\ \rho(x,y)<\delta\}$. Определение непрерывности тогда можно записать как
	\begin{equation}
		\forall\varepsilon>0\ \exists\delta\such\forall x\in U_{\delta}(x_0)\ f(x)\in U_{\varepsilon}(f(x_0)),
	\end{equation}
или, ещё короче,
	\begin{equation}
		\forall\varepsilon>0\ \exists\delta\such f(U_{\delta}(x_0))\subset U_{\varepsilon}(f(x_0)).
	\end{equation}
Повторим фокус снова: забудем про внутреннее устройство окрестности. Будем теперь считать, что каждой точке множества приписано семейство множеств, называемых окрестностями этой точки, свойства которых мы потом отдельно выделим. Максимально общо, определение непрерывности в точке теперь выглядит так, если мы предполагаем, что на множествах $X$ и $Y$ введены эти системы окрестностей.
\begin{defin}
	Функция $f\colon X\to Y$ непрерывна в точке $x_0\in X$, если
	\begin{equation}
		\forall U(f(x_0))\ \exists V(x_0)\such f(V(x_0))\subset U(f(x_0)).
	\end{equation}
\end{defin}

\subsection{Топологические конструкции}
\TODO{инишал и файнал топологии}

\TODO{индуцированная топология, фактортопология, действие группы на топологическом пространстве, универсальные свойства}

\TODO{тихоновская топология на произведении топологических пространств, теорема тихонова для конечного произведения (для бесконечного тоже ебани)}

\TODO{топологические операции, пространства с отмеченной точкой, букеты}

\subsection{Гомотопия и гомотопическая эквивалентность}
\begin{defin}
	\textit{Гомотопией} между отображениями $f,g\colon X\to Y$ называется отображение $H\colon X\times[0,1]\to Y$ такое, что $H|_{t=0}=f$ и $H|_{t=1}=g$, про сами отображения будем говорить что они \textit{гомотопны}.
\end{defin}
Если попытаться придать точный смысл \textit{деформации}, то получится гомотопия. \TODO{Объяснить это}

\subsection{Клеточные пространства и теорема о клеточной аппроксимации}

\subsection{Фундаментальная группа}
\TODO{теорема Брауэра о неподвижной точке, теорема Борсука-Улама}

\subsection{Теорема Зейферта-ван Кампена}
\TODO{свободное произведение}

\TODO{категорный смысл --- сохранение пушаутов}

\subsection{Фундаментальная группа клеточного пространства. Классификация двумерных поверхностей}

\subsection{Накрытия}
\TODO{свойство поднятия пути и гомотопии, универсальное накрытие, классификация накрытий, теорема Нильсена-Шраера}