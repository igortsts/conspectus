\section{Дифференциальная геометрия}
\subsection{Многообразия и гладкие отображения}
\begin{defin}
	Про хаусдорфово топологическое пространство $T$ будем говорить, что оно локально евклидово, или что оно является топологическим многоообразием размерности $\dim T=n$, если существует его покрытие открытыми множествами $\{U_{\alpha}\}$ и набор гомеоморфизмов $\phi_{\alpha}\colon U_{\alpha}\to \RR^n$.
\end{defin}
% TODO: обсуждение хаусдорфовости
Набор пар $\{(U_{\alpha},\phi_{\alpha})\}$ называется \textit{атласом} многообразия $T$, множества $U_{\alpha}$ --- \textit{картами}, а отображения $\phi_{\alpha}$ --- \textit{координатными гомеоморфизмами}. Отображения $x_{\alpha}^i=\pi_i\circ\phi_{\alpha}$ называют \textit{локальными координатами} на карте $U_{\alpha}$.

Анализ на многообразиях начинается с определения гладкости. Про атлас $\{(U_{\alpha},\phi_{\alpha})\}$ на топологическом многообразии $T$ будем говорить, что он принадлежит классу гладкости $C^k$, если для всякой пары карт $U_{\alpha}$, $U_{\beta}$ таких, что $U_{\alpha}\cap U_{\beta}\neq\varnothing$, \textit{функции перехода} $\varphi_{\alpha}\circ\varphi_{\beta}^{-1}\colon \phi_{\beta}(U_{\alpha}\cap U_{\beta})\to\phi_{\alpha}(U_{\alpha}\cap U_{\beta})$ --- гладкие функции класса $C^k$ для всех $\alpha$, $\beta$. Если функции перехода бесконечно гладкие, то будем говорить, что атлас гладкий.

На множестве атласов класса $C^k$ на многообразии $T$ можно ввести отношение эквивалентности: мы не будем различать два атласа, если их объединение --- снова атлас того же класса гладкости. Класс эквивалентности гладкого атласа будем называть \textit{гладкой структурой} на топологическом многообразии $M$.

Мы готовы дать главное определение этой главы.  
\begin{defin}
	Топологическое пространство $M$ называется гладким многообразием размерности $\dim M=n$ гладкости $C^k$, если оно хаусдорфово, обладает счётной локальной базой и является топологическим многообразием размерности $n$, наделённым гладкой структурой класса $C^k$.
\end{defin}
Требование существования счётной локальной базы пригодится в дальнейшем, когда мы будем определять разбиение единицы и интегралы на многообразии, это свойство окажется необходимым.

\begin{defin}
	Отображение $F\colon M\to R^m$ гладкое, если все отображения $F\circ\phi_{\alpha}\colon\RR^n\to\RR^m$ гладкие.
\end{defin}
Важный частный случай --- гладкие функции $f\colon M\to\RR$. Их множество образует алгебру $C^{\infty}(M)$.

Это определение естественно переносится на случай отображений между многообразиями.
\begin{defin}
	Отображение $F\colon M\to N$ гладкое, если для всяких карт $(U,\phi)$ на $M$ и $(V,\psi)$ на $N$ отображение $\psi\circ F\circ\phi^{-1}\colon \phi(U\cap F^{-1}(V))\to\psi(V)$ гладкое. 
\end{defin}

Определим частные производные функции $f\in C^{\infty}(M)$ в точке $p\in M$ как
\begin{equation}
 	\frac{\partial f}{dx^i}(p)=\frac{\partial(f\circ\phi^{-1})}{\partial x^i}(\phi(p)).
\end{equation}

Семейство гладких многообразий вместе с гладкими отображениями образует категорию гладких многообразий $\mathcal{M}an$.

\subsection{Касательное и кокасательное пространство}
Пусть $p\in M$. Определим \textit{дифференцирование} гладкой функции в точке $p$ как линейное отображение $X\colon C^{\infty}\to\RR$, удовлетворяющее правилу Лейбница: для любых функций $f,g\in C^{\infty}$ верно
\begin{equation*}
	X(fg)=X(f)g(p)+f(p)X(g).
\end{equation*}
Множество дифференцирований в точке $p$ будем обозначать $T_pM$. Введём на нём сложение и умножение на скаляр по правилам
\begin{eqnarray*}
	(X_1+X_2)(f)&=&X_1(f)+X_2(f),\\
	(\alpha X)(f)&=&\alpha X(f).
\end{eqnarray*}	
Таким образом, множество $T_pM$ --- линейное пространство. Его называют \textit{касательным пространством} к многообразию $M$ в точке $p$. Двойственное к нему пространство $(T_pM)^*$ называется \textit{кокасательным пространством} в точке $p$ и обозначается $T^*_pM$.

Сформулируем как отдельную лемму несложные свойства дифференцирований.
\begin{lem}
	Пусть $X\in T_pM$.
	\begin{enumerate}
		\item Если $f=\mathrm{const}$, то $Xf=0$.
		\item Если $f(p)=g(p)=0$, то $X(fg)=0$.
	\end{enumerate}
\end{lem}

% Пусть $(U,\phi)$ --- карта на $M$, $p\in U$, тогда $\frac{\partial}{\partial x^i}\in T_pM$ для $i\in\overline{0,n}$. Значит, любая линейная комбинация $X^i\frac{\partial}{\partial x^i}$ лежит в $T_pM$. Более того, касательное пространство ими исчерпывается, то есть $\left\{\frac{\partial}{\partial x^i}\right\}_1^n$ --- базис в $T_pM$.

Выясним, как выклядит касательное пространство к точке $p\in\RR^n$. Поскольку $\frac{\partial}{\partial x^i}\in T_p\RR^n$ для $i\in\overline{0,n}$, то любая линейная комбинация $X^i\cfrac{\partial}{\partial x^i}$ лежит в $T_p\RR^n$. Мы построили оператор $v\colon\RR^n\to T_p\RR^n$, $X=(X^1,\ldots,X^n)\mapsto X^i\cfrac{\partial}{\partial x^i}$. Покажем, что он является изоморфизмом. В самом деле, $v$ инъективен, так как если $v(X)=0$, то для всякой функции $v(X)(f)=0$, в том числе и для всех координатных функций $x^j$:
\begin{equation}
	0=X^i\frac{\partial x^j}{\partial x^i}(p)=X^j,
\end{equation}
значит, $X=0$. Чтобы показать сюръективность, вспомним, что всякая гладкая функция раскладывается по формуле Тейлора в точке $p$ как
\begin{equation*}
	f(x)=f(p)+\sum_{i=1}^n\frac{\partial f}{\partial x^i}(p)(x^i-p^i)+\sum_{i=1}^ng_i(x)(x^i-p^i),
\end{equation*}
где функции $g_i$ гладкие и $g_i(p)=0$. Выберем $X\in T_p M$ и обозначим $X^i=X(x^i)$, тогда
\begin{eqnarray*}
	Xf&=&X(f(a))+\sum_{i=1}^n\frac{\partial f}{\partial x^i}(p)(X(x^i)-X(p^i))+\sum_{i=1}^{n}X(g_i(x)(x^i-p^i))\\
		&=&\left(\sum_{i=1}^{n} X^i\frac{\partial}{\partial x^i}\Big|_p\right)f.
\end{eqnarray*}	
Мы доказали, что дифференцирования $\left\{\frac{\partial}{\partial x^i}\right\}_1^n$ вдоль координатных линий образуют базис в пространстве $T_p\RR^n$ и что тем самым его размерность равна $n$.

Выберем точку $p\in M$. Определим \textit{дифференциал в точке} (push-forward) как функтор, каждому многообразию $M$ ставящий касательное пространство $T_pM$, а гладкому отображению $F\colon M\to N$ линейный оператор $F_*\colon T_pM\to T_{F(p)}M$, действующий по правилу
\begin{equation}
	(F_*X)(f)=X(F\circ f).
\end{equation}
Убедимся, что $F_*$ корректно определён, то есть что он линеен, и $F_*X$ действительно дифференцирование для любых $X\in T_pM$.
% убеждаемся
Несложно убедиться, что это действительно функтор. В дальнейшем мы усложним конструкцию, перенеся её на касательные расслоения.
\begin{lem}
	Для всякой окрестности $U$ точки $p\in M$ верно $T_pU=T_pM$.
\end{lem}
\begin{proof}

\end{proof}
В силу функториальности дифференциала имеем $T_pU=T_\phi(p)\RR^n$, тогда $T_pM=T_{\phi(p)}\RR^n$.

\subsection{Касательное и кокасательное расслоение}
\subsection{Тензорное поле на многообразии}